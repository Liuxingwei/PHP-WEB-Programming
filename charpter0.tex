\chapter{第〇章 Web 编程概述}

\section{网页、HTML和浏览器}

一句话总结{\新楷体\textbf{网页}}、{\新楷体\textbf{HTML}}和{\新楷体\textbf{浏览器}}之间的关系:就是{\新楷体\textbf{网页是用HTML编写的,要将HTML语言编写的文件呈献为预定的外观,需要交由浏览器进行解析和渲染}}。

接下来,我们将通过自己动手制作简单的网页来提示其中的内在联系。

\subsection{自己动手做网页}

1. 打开 vscode

2. 新建文件

3. 输入如下内容

\begin{verbatim}
    <!DOCTYPE html>
    <html lang="zh-CN">
    <head>
        <meta charset="UTF-8">
        <title>第一个网页</title>
    </head>
    <body>
        <h4 title="我是一个快乐的web程序员">人生中的第一个网页</h4>
        <hr>
        <p>稍微简单了点,哈哈!</p>
    </body>
    </html>
\end{verbatim}

4. 保存文件为 hello.html

5. 将保存的文件拖入浏览器窗口

大功告成!

\subsection{HTML - 超文本标记语言}

上面输入的内容,就是网页的原始文本,它是使用 HTML (超文本标记语言)编写的。

超文本是指文本中可以包含图片、链接、多媒体、程序等非文本资源,而文本中的内容呈现是用标记指定的,浏览器通过解析这些标记来渲染网页。

HTML的标记被定义成一系列的标签,每个标签有特定的主义。标签写在一对尖括号内,不区分大小写。<html lang=''zh-CN''>、<head>、<body>等都是标答。

HTML是以标记作为语言的特征,但其基本单位是元素(Element)。一个完整的元素包含起始标签、标签内容、结束标签,起始标签还会有属性。

上文的“\verb|<h4 title="我是一个快乐的web程序员">人生中的第一个网页</h4>|”就是一个完整的元素。

“\verb|<h4 title="我是一个快乐的web程序员">|”是起始标签,其中“\verb|title="我是一个快乐的web程序员"|”是标签属性。

“\verb|</h4>|”是结束标签。

“\verb|人生中的第一个网页|”是标签内容。

元素中只有开始标签是必须的,其他部分--结束标签、元素内容、元素属性--都是可选的。

“\verb|<p>稍微简单了点,哈哈!</p>|”没有元素属性。“\verb|<meta charset="UTF-8">|”没有结束标签和元素内容。“\verb|<hr>|”只有起始标签。

HTML 的详细讲解见【第一部分 HTML】。

\subsection{浏览器}

上网总是离不开浏览器,作为互联网最重要的工具软件,网页魔法都是由浏览器这个魔法师实现的。

浏览器的第一个功能是解析和渲染HTML:

在第一小节中,当我们把编写好的 HTML 文件拖入浏览器,天书一般的文本就呈现为简洁的网页了。

例子中,元素“\verb|<title>第一个网页</title>|”的元素内容“\verb|第一个网页|”,显示在浏览器标签页标题中。

元素“\verb|<h4 title="我是一个快乐的web程序员">人生中的第一个网页</h4>|”的元素内容“\verb|人生中的第一个网页|”以粗体4号标题显示在网页中,元素属性“\verb|title="我是一个快乐的web程序员"|”不直接显示,当鼠标悬停在这个元素上时,以提示框的形式显示“\verb|我是一个快乐的web程序员|”。

浏览器的第二个功能是与服务器交互,向服务器发送请求,接收服务器返回的结果。(见「服务器与客户端」小节)

浏览器的第三个功能是运行客户端脚本,主要是JavaScript脚本,动态改变网页的显示,增强与用户的交互体验。(见「客户端编程」小节)

\section{服务器与客户端}



\section{静态网页和动态网页}

\section{Web 编程语言}

\section{数据库}

\section{客户端编程}